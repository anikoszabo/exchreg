\documentclass[hyperref={bookmarks=false},aspectratio=169]{beamer}
\usepackage[utf8]{inputenc}
\usepackage{amsmath}
\usepackage{silence,lmodern}
\usepackage[style=verbose,backend=biber]{biblatex}
%\addbibresource{./figures/vv.bib}
% Filter warnings issued by package biblatex starting with "Patching footnotes failed"
\WarningFilter{biblatex}{Patching footnotes failed}



% ---------------  Define theme and color scheme  -----------------
\usetheme[sidebarleft]{Caltech}  % 3 options: minimal, sidebarleft, sidebarright



% ------------  Information on the title page  --------------------
\title[Semi-parametric model]
{\bfseries{Semi-parametric Model for Clustered Binary Outcome}}

\author[Aniko Szabo, Xinran Qi]
{Aniko Szabo\& Xinran Qi}

\institute[MCW]
{Medical College of Wisconsin}

\date[October 11, 2019]
{\footnotesize ICOSDA 2019}
%------------------------------------------------------------

%------------------------------------------------------------
%The next block of commands puts the table of contents at the
%beginning of each section and highlights the current section:

\AtBeginSection[]
{
  \begin{frame}
    \frametitle{Table of Contents}
    \tableofcontents[currentsection]
  \end{frame}
}
%------------------------------------------------------------



\begin{document}

\frame{\titlepage}  % Creates title page

%---------   table of contents after title page  ------------
\begin{frame}
\frametitle{Table of Contents}
\tableofcontents
\end{frame}
%---------------------------------------------------------



\section{Introduction}

%---------------------------------------------------------
%Two columns
\begin{frame}
\frametitle{Developmental toxicity of boron acid (BA) \footcite{heindel1994developmental}}

\begin{columns}

\column{0.6\textwidth}

\begin{figure}
    \centering
    \includegraphics[width=\columnwidth]{./figures/Rat.png}
    \caption{Developmental toxicity experiment of boron acid}
    \label{fig:ratExperiment}
\end{figure}

\column{0.4\textwidth}
\begin{itemize}
    \item National Toxicology Program provided pregnant mice with feed containing BA at different dose levels.
    \item Number of fetuses in litter NOT affected by dose exposure.
    \item Number of resorptions, deaths, or other abnormalities.
\end{itemize}

\end{columns}
\end{frame}
%---------------------------------------------------------

%---------------------------------------------------------
%Two columns
\begin{frame}
\frametitle{Developmental toxicity of boron acid (BA) Cont.}

\begin{columns}

\column{0.5\textwidth}

\begin{figure}
    \centering
    \includegraphics[width=\columnwidth]{./figures/RatBarplot.png}
    \caption{Bar plot for developmental toxicity experiment of boron acid}
    \label{fig:ratExperimentBarplot}
\end{figure}

\column{0.5\textwidth}
\begin{itemize}
    \item Primary parameter of interest: entire distribution for combined endpoints from varying cluster sizes {\&} under different dose levels exposure.
    \item Correlated fetuses: gestate in same litter {\&} exposed to same/different concentrations of BA.
    \item Model within-cluster effects {\&} exogenous risks: second/higher order correlation coefficients as parameterization.
\end{itemize}

\end{columns}
\end{frame}
%---------------------------------------------------------

%---------------------------------------------------------
%Two columns
\begin{frame}
\frametitle{Developmental toxicity of boron acid (BA) Cont.}

\begin{columns}

\column{0.5\textwidth}

\begin{figure}
    \centering
    \includegraphics[width=\columnwidth]{./figures/RatBarplotRatAdded.png}
    \caption{Bar plot for developmental toxicity experiment of boron acid}
    \label{fig:ratExperimentBarplotRatAdded}
\end{figure}

\column{0.5\textwidth}
\begin{itemize}
    \item Primary parameter of interest: entire distribution for combined endpoints from varying cluster sizes {\&} under different dose levels exposure.
    \item Correlated fetuses: gestate in same litter {\&} exposed to same/different concentrations of BA.
    \item Model within-cluster effects {\&} exogenous risks: second/higher order correlation coefficients as parameterization.
\end{itemize}

\end{columns}
\end{frame}
%---------------------------------------------------------



\section{Clustered Binary Outcomes}

%---------------------------------------------------------
%Two columns
\begin{frame}
\frametitle{Notations for Clustered Binary Outcomes}

\begin{columns}

\column{0.38\textwidth}

\begin{figure}
    \centering
    \includegraphics[width=\columnwidth]{./figures/notations.png}
    \caption{Illustration of notations for clustered binary outcomes exposed to single dose level}
    \label{fig:notations}
\end{figure}

\column{0.62\textwidth}
\begin{itemize}
    \item A sample of clusters $(i = 1, \dots, I)$ with cluster sizes $(N_1, \dots, N_i, \dots, N_I)$.
    \item $\left\lbrace X_{i1}, \dots, X_{ij}, \dots, X_{iN_i} \right\rbrace$, $j \in \left\lbrace 1, \dots, N_i \right\rbrace$: observations in $i^{th}$ cluster of size $N_i$.
    \item $X_{ij}$: binary outcome, 1 indicating success {\&} 0 indicating failure.
    \item $Y_i = \sum_{j=1}^{N_i} X_{ij}$: total number of successes from cluster of size $N_i$.
    \item $q_{y,N} = \textbf{Pr}_N(\sum_{j=1}^{N} X_j = y)$: marginal probability of achieving $y$ successes from cluster of size $N$.
    \item $\mathcal{N} \geq \max{(N_{i})}$, where $1 \leq i \leq I$: maximal possible cluster size.
\end{itemize}

\end{columns}
\end{frame}
%---------------------------------------------------------

%---------------------------------------------------------
%Changing visivility of the text
\begin{frame}
\frametitle{Modeling Exchangeable Clustered Binary Outcomes}

\begin{block}{Assumption: Exchangeability}
A sequence of binary random variables $\left\lbrace X_1, X_2, \dots, X_N \right \rbrace$ being exchangeable if for any $N$, any vector $(x_1, x_2, \dots, x_N)'$ and any permutation of indexes $\left\lbrace 1, 2, \dots, N \right\rbrace$:

$\textbf{Pr}(X_{\varphi(1)} = x_1, \dots, X_{\varphi(N)} = x_N) = \textbf{Pr}(X_1=x_1, \dots, X_N=x_N)$.
\end{block}

\textbf{Fixed or random effect models}

\begin{itemize}
    \item Fixed effect: Tallis [1962] added additional parameter $\rho$ to probability generating functions {\&} derived joint probability distribution of correlated observations within same cluster \footcite{tallis1962use}.
    \item Random effect: Williams [1975] proposed beta-binomial model assuming binary responses subject to Bernoulli whose probability varied among clusters according to Beta distribution \footcite{WilliamsDA1975}.
\end{itemize}

\end{frame}
%---------------------------------------------------------

%---------------------------------------------------------
%Changing visivility of the text
\begin{frame}
\frametitle{Modeling Exchangeable Clustered Binary Outcomes Cont.}

\textbf{Generalized estimating equation (GEE) models}

\begin{itemize}
    \item Expectation {\&} covariance matrix obtained using properties of generalized exponential family/direct derivation from given distributions {\&} definitions.
    \item Robust estimates of regression coefficients under misspecified working correlation matrix.
    \item Research objectives: marginal mean {\&} pairwise correlation \footcite{PrenticeZhao1991}, odds ratio \footcite{lipsitz1991generalized}, mean response {\&} intra-cluster correlation \footcite{BowmanChen1995}.
\end{itemize}

\end{frame}
%---------------------------------------------------------

%---------------------------------------------------------
%Changing visivility of the text
\begin{frame}
\frametitle{Modeling Exchangeable Clustered Binary Outcomes Cont.}

\textbf{Higher order moments and correlations}

\begin{itemize}
    \item Bowman and George [1995] proposed to use joint probability $\left\lbrace \lambda_k, k = 1, \dots, N \right\rbrace$ to reparametrize likelihood function \footcite{bowman1995saturated}.
    \item $\lambda_k = \textbf{Pr}(X_{i_1} = X_{i_2} = \dots = X_{i_k} = 1)$, with $\lambda_0=1$, $\lambda_1 = \textbf{Pr} (X_1 = 1)$, $\lambda_2 = \textbf{Pr} (X_1 = X_2 = 1)$.
\end{itemize}

\begin{block}{1-1 Mapping: Joint probability $\left\lbrace \lambda_k \right\rbrace$ {\&} marginal probability $\left\lbrace q_y \right\rbrace$}
\begin{equation*}\label{E:1to1}
q_y = \binom{N}{y} \sum_{k=0}^{N-y} {(-1)^k} \binom{N-y}{k} \lambda_{y+k} ,
\end{equation*}

\begin{equation*}\label{E:also1to1}
\lambda_k = \sum_{y=0}^{N-k} \frac{\binom{N-k}{y} {q_{N-y}}}{\binom{N}{N-y}} , y = 0, 1, \dots, N.
\end{equation*}
\end{block}

\end{frame}
%---------------------------------------------------------

%---------------------------------------------------------
%Changing visivility of the text
\begin{frame}
\frametitle{Modeling Exchangeable Clustered Binary Outcomes Cont.}

\textbf{Higher order moments and correlations}

\begin{itemize}
    \item A sequence of $\lambda$'s define valid distribution $\Longleftrightarrow$ completely monotone \footcite{bowman1995saturated}.
\end{itemize}

\begin{block}{Definition: Forward differences in $\left\lbrace \lambda_k \right\rbrace$}
\begin{equation*}\label{D:CM}
\Delta^{N-k} (\lambda_k) \geq 0, k = 0, \dots, N,
\end{equation*}
where ${\Delta}(\lambda_k) = \lambda_k - \lambda_{k+1}, {\Delta^2}(\lambda_k) = {\Delta}(\lambda_k) - {\Delta}(\lambda_{k+1})$, etc.
\end{block}

\end{frame}
%---------------------------------------------------------

%---------------------------------------------------------
%Changing visivility of the text
\begin{frame}
\frametitle{Modeling Exchangeable Clustered Binary Outcomes Cont.}

\textbf{Higher order moments and correlations}

\begin{itemize}
    \item Stefanescu and Turnbull [2003] proposed additional assumption “marginal compatibility” to estimate  $\left\lbrace \lambda_k \right\rbrace$ under varying cluster sizes \footcite{stefanescu2003likelihood}.
    \item Pang and Kuk [2007] proposed penalized kernel smoothing method which performed smoothing in both covariate and response space \footcite{pang2007test}.
\end{itemize}

\begin{block}{Assumption: Marginal compatibility}
For a cluster of size $N_1$ $(< N_2)$,
\begin{equation*}\label{A:marginalCompatibility}
\begin{split}
\lambda_{k,N_1} &= \boldsymbol{\Pr} \left( X_{1,N_1} = \dots = X_{k,N_1} = 1 \right) \\
                &= \boldsymbol{\Pr} \left( X_{1,N_2} = \dots = X_{k,N_2} = 1\right) = \lambda_{k,N_2}.
\end{split}
\end{equation*}
\end{block}

\end{frame}
%---------------------------------------------------------



\section{Clustered Multinomial Outcomes}

%---------------------------------------------------------
%Two columns
\begin{frame}
\frametitle{Example of Clustered Multinomial Outcomes}

\begin{columns}

\column{0.35\textwidth}

\begin{figure}
    \centering
    \includegraphics[width=\columnwidth]{./figures/DMF1.png}
    \caption{Illustration of DMF \newline \tiny{(Pictures downloaded from: https://www.vectorstock.com/royalty-free-vector)}}
    \label{fig:toothDMF}
\end{figure}

\column{0.65\textwidth}
\begin{itemize}
    \item NCHS conducted periodic survey, National Health and Nutrition Examination Survey (NHANES), based on complex, multi-stage sample design.
    \item Provide national estimates of health {\&} nutritional status of US civilian, non-institutionalized population aged two months and older.
    \item DMF Index: total number of teeth/surfaces decayed, missing, or filled in individual.
    \item DMFT index: applied to teeth, scores per individual range from 0 to 28/32.
\end{itemize}

\end{columns}
\end{frame}
%---------------------------------------------------------

%---------------------------------------------------------
%Two columns
\begin{frame}
\frametitle{Example of Clustered Multinomial Outcomes}

\begin{columns}

\column{0.35\textwidth}

\begin{figure}
    \centering
    \includegraphics[width=\columnwidth]{./figures/DMF3pics.png}
    \caption{Illustration of DMF \newline \tiny{(Pictures downloaded from: https://www.vectorstock.com/royalty-free-vector)}}
    \label{fig:toothDMF3Left}
\end{figure}

\column{0.65\textwidth}
\begin{itemize}
    \item NCHS conducted periodic survey, National Health and Nutrition Examination Survey (NHANES), based on complex, multi-stage sample design.
    \item Provide national estimates of health {\&} nutritional status of US civilian, non-institutionalized population aged two months and older.
    \item DMF Index: total number of teeth/surfaces decayed, missing, or filled in individual.
    \item DMFT index: applied to teeth, scores per individual range from 0 to 28/32.
\end{itemize}

\end{columns}
\end{frame}
%---------------------------------------------------------

%---------------------------------------------------------
%Two columns
\begin{frame}
\frametitle{Example of Clustered Multinomial Outcomes}

\begin{columns}

\column{0.35\textwidth}

\begin{figure}
    \centering
    \includegraphics[width=\columnwidth]{./figures/DMF2pics.png}
    \caption{Illustration of DMF \newline \tiny{(Pictures downloaded from: https://www.vectorstock.com/royalty-free-vector)}}
    \label{fig:toothDMF2Left}
\end{figure}

\column{0.65\textwidth}
\begin{itemize}
    \item NCHS conducted periodic survey, National Health and Nutrition Examination Survey (NHANES), based on complex, multi-stage sample design.
    \item Provide national estimates of health {\&} nutritional status of US civilian, non-institutionalized population aged two months and older.
    \item DMF Index: total number of teeth/surfaces decayed, missing, or filled in individual.
    \item DMFT index: applied to teeth, scores per individual range from 0 to 28/32.
\end{itemize}

\end{columns}
\end{frame}
%---------------------------------------------------------

%---------------------------------------------------------
%Two columns
\begin{frame}
\frametitle{Example of Clustered Multinomial Outcomes}

\begin{columns}

\column{0.35\textwidth}

\begin{figure}
    \centering
    \includegraphics[width=\columnwidth]{./figures/DMF1pics.png}
    \caption{Illustration of DMF \newline \tiny{(Pictures downloaded from: https://www.vectorstock.com/royalty-free-vector)}}
    \label{fig:toothDMF1Left}
\end{figure}

\column{0.65\textwidth}
\begin{itemize}
    \item NCHS conducted periodic survey, National Health and Nutrition Examination Survey (NHANES), based on complex, multi-stage sample design.
    \item Provide national estimates of health {\&} nutritional status of US civilian, non-institutionalized population aged two months and older.
    \item DMF Index: total number of teeth/surfaces decayed, missing, or filled in individual.
    \item DMFT index: applied to teeth, scores per individual range from 0 to 28/32.
\end{itemize}

\end{columns}
\end{frame}
%---------------------------------------------------------



\section{Exponential Tilting}

%---------------------------------------------------------
%Changing visivility of the text
\begin{frame}
\frametitle{Definition of Exponential Tilting}

\begin{columns}

\column{0.4\textwidth}

\begin{figure}
    \centering
    \includegraphics[width=\columnwidth]{./figures/p.png}
    \caption{Original probability distribution}
    \label{fig:OriginalPMF}
\end{figure}

\column{0.6\textwidth}
\begin{itemize}
    \item For random variable $Y$ with probability density distribution $\boldsymbol{p}$ {\&} mean at $\mu_0$.
    \item New distribution $\boldsymbol{f}$ with shortest Kullback - Leibler distance/divergence to distribution $\boldsymbol{p}$ {\&} mean at pre-specified value $\mu_1$.
    \item $f_y \propto p_y  \exp (\theta y)$: natural exponential family generated from distribution $\boldsymbol{p}$ via exponential tilting \footcite{EETCarloCamarda}.
    \item $\Longleftrightarrow$ distribution $\boldsymbol{p}$ with Lagrange multipliers for constraints define distribution $\boldsymbol{f}$.
\end{itemize}

\end{columns}
\end{frame}
%---------------------------------------------------------

%---------------------------------------------------------
%Changing visivility of the text
\begin{frame}
\frametitle{Definition of Exponential Tilting}

\begin{columns}

\column{0.4\textwidth}

\begin{figure}
    \centering
    \includegraphics[width=\columnwidth]{./figures/pf.png}
    \caption{Original and exponentially tilted probability distributions}
    \label{fig:OriginalETPMF1}
\end{figure}

\column{0.6\textwidth}
\begin{itemize}
    \item For random variable $Y$ with probability density distribution $\boldsymbol{p}$ {\&} mean at $\mu_0$.
    \item New distribution $\boldsymbol{f}$ with shortest Kullback - Leibler distance/divergence to distribution $\boldsymbol{p}$ {\&} mean at pre-specified value $\mu_1$.
    \item $f_y \propto p_y  \exp (\theta y)$: natural exponential family generated from distribution $\boldsymbol{p}$ via exponential tilting \footcite{EETCarloCamarda}.
    \item $\Longleftrightarrow$ distribution $\boldsymbol{p}$ with Lagrange multipliers for constraints define distribution $\boldsymbol{f}$.
\end{itemize}

\end{columns}
\end{frame}
%---------------------------------------------------------

%---------------------------------------------------------
%Changing visivility of the text
\begin{frame}
\frametitle{Definition of Exponential Tilting}

\begin{columns}

\column{0.4\textwidth}

\begin{figure}
    \centering
    \includegraphics[width=\columnwidth]{./figures/pff.png}
    \caption{Original and exponentially tilted probability distributions}
    \label{fig:OriginalETPMF2}
\end{figure}

\column{0.6\textwidth}
\begin{itemize}
    \item For random variable $Y$ with probability density distribution $\boldsymbol{p}$ {\&} mean at $\mu_0$.
    \item New distribution $\boldsymbol{f}$ with shortest Kullback - Leibler distance/divergence to distribution $\boldsymbol{p}$ {\&} mean at pre-specified value $\mu_1$.
    \item $f_y \propto p_y  \exp (\theta y)$: natural exponential family generated from distribution $\boldsymbol{p}$ via exponential tilting \footcite{EETCarloCamarda}.
    \item $\Longleftrightarrow$ distribution $\boldsymbol{p}$ with Lagrange multipliers for constraints define distribution $\boldsymbol{f}$.
\end{itemize}

\end{columns}
\end{frame}
%---------------------------------------------------------

%---------------------------------------------------------
%Changing visivility of the text
\begin{frame}
\frametitle{Generalized Linear Models with Unspecified Reference Distributions for Continuous Outcomes \footcite{rathouz2008generalized}}

\begin{columns}

\column{0.4\textwidth}

\begin{figure}
    \centering
    \includegraphics[width=\columnwidth]{./figures/orginalModel.png}
    \caption{Flowchart of Rathouz and Gao's semi-parametric generalized linear model}
    \label{fig:RathouzGaoSPGLM}
\end{figure}

\column{0.6\textwidth}
\begin{itemize}
    \item Rathouz and Gao [2008] proposed semi-parametric generalized linear model, specified with linear predictor {\&} link function for mean response $Y$.
    \item \alert{Conditional mean model}: $E\left( Y|\boldsymbol{Z};\boldsymbol{\beta} \right) =\mu\left( \boldsymbol{Z}, \boldsymbol{\beta} \right) \equiv \mu = h^{-1}\left( \eta\right)$, where $\eta = \boldsymbol{Z}' \boldsymbol{\beta}$.
    \item \alert{Probability density}: $f\left( y |\boldsymbol{Z};\boldsymbol{\beta}, f_0\right) \propto f_0\left( y\right) \exp(\theta y)$.
    \item Extend framework by estimating $f_0$ from data {\&} maintaining mean model at user-specified $h\left( \mu\right) =\eta$.
\end{itemize}

\end{columns}
\end{frame}
%---------------------------------------------------------



\section{Research Aims for Dissertation}

%---------------------------------------------------------
%Changing visivility of the text
\begin{frame}
\frametitle{Aim 1: Develop A Semi-parametric Generalized Linear Model (SPGLM) for Clustered Binary Outcomes}

\begin{columns}

\column{0.4\textwidth}

\begin{figure}
    \centering
    \includegraphics[width=\columnwidth]{./figures/modelAim1.png}
    \caption{Flowchart of extending SPGLM to model clustered binary outcomes with varying cluster sizes}
    \label{fig:Aim11}
\end{figure}

\column{0.6\textwidth}
\begin{itemize}
    \item Extend SPGLM to model clustered binary outcomes with varying cluster sizes.
    \item Response variable: number of successes $Y$ scaled by cluster size $N$.
    \item \alert{Assumption}: probability density of $Y_g/N$ exposed to $g^{th}$ ($g = 1, \dots, G$) dose level, $\left\lbrace q_{y,N}^{(g)} \right\rbrace$, is exponentially tilted version of reference dose level's density $\left\lbrace q_{y,N} \right\rbrace$.
    \item \alert{Conditional mean model}: \newline
    $E\left( \frac{Y_g}{N} | \boldsymbol{Z}_g, \boldsymbol{\beta}_g \right) = \mu(\boldsymbol{Z}_g, \boldsymbol{\beta}_g)  \equiv \mu_g = h^{-1} \left( \boldsymbol{Z}_g^{'} \boldsymbol{\beta}_g \right)$, where $\mu_g \in (0, 1)$.
\end{itemize}

\end{columns}
\end{frame}
%---------------------------------------------------------

%---------------------------------------------------------
%Changing visivility of the text
\begin{frame}
\frametitle{Aim 1: Develop A Semi-parametric Generalized Linear Model (SPGLM) for Clustered Binary Outcomes Cont.}

\begin{columns}

\column{0.4\textwidth}

\begin{figure}
    \centering
    \includegraphics[width=\columnwidth]{./figures/modelAim1.png}
    \caption{Flowchart of extending SPGLM to model clustered binary outcomes with varying cluster sizes}
    \label{fig:Aim12}
\end{figure}

\column{0.6\textwidth}
\begin{itemize}
    \item \alert{Probability density}: for a cluster of size $N$, $ q_{y,N}^{(g)} \left( \boldsymbol{Z}_g, \boldsymbol{\beta}_g \right) \propto q_{y,N} \times \exp{ \left[ - \omega_g(\boldsymbol{Z}_g, \boldsymbol{\beta}_g) y \right]} $.
    \item $\omega_g(\boldsymbol{Z}_g, \boldsymbol{\beta}_g)$: tilting parameter for $g^{th}$ dose level ($\omega_0 \equiv 0$ for reference group), $\Longrightarrow$ $\displaystyle \sum_{y=0}^N \dfrac{y q_{y,N}^{(g)}}{N} = \mu_g = h^{-1}\left( \boldsymbol{Z}_g^{'} \boldsymbol{\beta}_g \right)$.
    \item For identifiability, mean of $\left\lbrace q_{y,N}\right\rbrace$ equal to arbitrary fixed value.
    \item \alert{Extend models for varying cluster sizes by assuming marginal compatibility}.
\end{itemize}

\end{columns}
\end{frame}
%---------------------------------------------------------

%---------------------------------------------------------
%Changing visivility of the text
\begin{frame}
\frametitle{Aim 1: Develop A Semi-parametric Generalized Linear Model (SPGLM) for Clustered Binary Outcomes Cont.}

\begin{columns}

\column{0.4\textwidth}

\begin{figure}
    \centering
    \includegraphics[width=\columnwidth]{./figures/modelAim1.png}
    \caption{Flowchart of extending SPGLM to model clustered binary outcomes with varying cluster sizes}
    \label{fig:Aim13}
\end{figure}

\column{0.6\textwidth}
\begin{itemize}
    \item Similar Expectation Maximization Newton-Raphson algorithm proposed by Rathouz and Gao [2008] for parameter estimation.
    \item Increase {\&} decrease risk from reference probability density.
    \item Application: BA rat data.
    \item Simulation studies: finite-sample properties {\&} compare results with existing methods.
\end{itemize}

\end{columns}
\end{frame}
%---------------------------------------------------------

%---------------------------------------------------------
%Changing visivility of the text
\begin{frame}
\frametitle{Aim 2: Generalize Concept of Marginal Compatibility for Clustered Binary Outcomes}

\textbf{Parametric extension}: include certain transformation of cluster size as covariate in linear predictor.

\begin{columns}

\column{0.4\textwidth}

\begin{figure}
    \centering
    \includegraphics[width=\columnwidth]{./figures/modelAim2.png}
    \caption{Flowchart of generalizing marginal compatibility for clustered binary outcomes}
    \label{fig:Aim21}
\end{figure}

\column{0.6\textwidth}
\begin{itemize}
    \item \alert{Parametric conditional mean model}:
    \begin{equation*}\label{M:parametricSPGLM_GMCconditionalMean}
    \begin{split}
    E\left( \frac{Y_i}{N_i} | \boldsymbol{Z}_i, \boldsymbol{\beta} \right) &= \mu(\boldsymbol{Z}_i, \boldsymbol{\beta})  \equiv \mu_i  \\
    &= h^{-1} \left( \boldsymbol{Z}_i^{'} \boldsymbol{\beta} \right) \\
    &= h^{-1} \left( \beta_0 + \beta_1 \log(N_i) \right) . \\
    \end{split}
    \end{equation*}
    \item Cluster size included in $\boldsymbol{Z}$ as covariate, $\boldsymbol{Z}_i = \left( 1, \log(N_i) \right)^{'}$.
\end{itemize}

\end{columns}
\end{frame}
%---------------------------------------------------------

%---------------------------------------------------------
%Changing visivility of the text
\begin{frame}
\frametitle{Aim 2: Generalize Concept of Marginal Compatibility for Clustered Binary Outcomes Cont.}

\textbf{Parametric extension}: include certain transformation of cluster size as covariate in linear predictor.

\begin{columns}

\column{0.4\textwidth}

\begin{figure}
    \centering
    \includegraphics[width=\columnwidth]{./figures/modelAim2.png}
    \caption{Flowchart of generalizing marginal compatibility for clustered binary outcomes}
    \label{fig:Aim22}
\end{figure}

\column{0.6\textwidth}
\begin{itemize}
    \item For a cluster of size $N_i$ ($\leq N$), \newline
    $q_{Y_i,N_i} =\binom{N_i}{Y_i} \displaystyle \sum_{t=Y_i}^{N-N_i+Y_i} \dfrac{\binom{N-N_i}{t-Y_i}} {\binom{N}{t}}$\alert{$q_{t,N|N_i}$}, $Y_i = 0, \dots, N_i$, $t=0, \dots, N$.
    \item \alert{Parametric generalized marginal compatibility model}: \newline
    $q_{t,N|N_i}\left( \boldsymbol{Z}_i, \boldsymbol{\beta} \right) \propto q_{t,N} \times \exp{ \left[ - \omega(\boldsymbol{Z}_i, \boldsymbol{\beta}) t \right]} $.
\end{itemize}

\end{columns}
\end{frame}
%---------------------------------------------------------

%---------------------------------------------------------
%Changing visivility of the text
\begin{frame}
\frametitle{Aim 2: Generalize Concept of Marginal Compatibility for Clustered Binary Outcomes Cont.}

\textbf{Non-parametric extension}: assign different tilting parameters to different cluster sizes.

\begin{itemize}
    \item \alert{Assumption}: for smaller cluster size $N'$ ($<N$), probability density $\left\lbrace q_{y,N'}, y = 0, \dots, N' \right\rbrace$, subject to \alert{marginal compatibility} {\&} derive from exponentially tilted version of maximum cluster size's density $\left \lbrace q_{t,N|N'}, t=0,\dots, N\right\rbrace$.
    \item $q_{y,N'} =\binom{N'}{y} \displaystyle \sum_{t=y}^{N-N'+y} \dfrac{\binom{N-N'}{t-y} q_{t,N|N'}}{\binom{N}{t}}$.
    \item \alert{Non-parametric conditional mean model}:
    \begin{equation*}\label{M:NONparametricSPGLM_GMCconditionalMean}
    E\left( \frac{Y}{N} | N', \boldsymbol{Z}, \boldsymbol{\beta} \right) = \mu(N', \boldsymbol{Z}, \boldsymbol{\beta})  \equiv \mu = h^{-1} \left( \boldsymbol{Z}^{'} \boldsymbol{\beta} \right) .
    \end{equation*}
\end{itemize}

\end{frame}
%---------------------------------------------------------

%---------------------------------------------------------
%Changing visivility of the text
\begin{frame}
\frametitle{Aim 2: Generalize Concept of Marginal Compatibility for Clustered Binary Outcomes Cont.}

\textbf{Non-parametric extension}: assign different tilting parameters to different cluster sizes.

\begin{itemize}
    \item \alert{Non-parametric generalized marginal compatibility model}:
    \begin{equation*}\label{M:NONparametricSPGLM_GMCexponentialTiltPDF}
    \begin{split}
    q_{y,N|N'} &= q_{y,N} \times \exp{ \left[ - \omega_{N'} y - b \left( \omega_{N'} \right) \right]} \\
    & \propto q_{y,N} \times \exp{ \left( - \omega_{N'} y \right)} .
    \end{split}
    \end{equation*}
    \item Similar Expectation Maximization Newton-Raphson algorithm \footcite{rathouz2008generalized} for parameter estimation.
    \item Additional penalty $\varrho \displaystyle\sum_{N'=1}^N (\omega_{N'} - \omega_{N'-1})^2$, $\omega_N=0$, assigned in Lagrangian.
\end{itemize}

\end{frame}
%---------------------------------------------------------

%---------------------------------------------------------
%Changing visivility of the text
\begin{frame}
\frametitle{Aim 3: Develop A Semi-parametric Model for Clustered Ordinal Outcomes}

\alert{Semi-parametric model}:
\begin{equation*}\label{M:SPforOrdinal1}
\begin{split}
    \boldsymbol{\Pr} \big( & X_1 = \dots = X_{C_1} = 1,  X_{C_1 + 1} = \dots = X_{C_1 + C_2} = 2, \cdots, \\
   & X_{C_1 + \dots + C_{a-1} + 1} = \dots = X_{C_1 + \cdots + C_{a-1} +  C_{a}} = a, \cdots, \\
   & X_{C_1 + \dots + C_{A-2} + 1} = \dots = X_{C_1 + \dots + C_{A-2} +  C_{A-1}} = A-1, \\
   & X_{C_1 + \dots + C_{A-1} + 1} = \dots = X_{C_1 + \dots + C_{A-1} +  C_{A}} = A | N \big) \\
   & {\color{white} = {\boldsymbol{\Pr}}^{(1)} \big( X_1 = \dots = X_{C_1} = 1 | N \big) \times {\boldsymbol{\Pr}}^{(2)} \big(X_{1} = \dots = X_{C_2} = 2 | N - C_1\big) \times \cdots }\\
   & {\color{white} \times {\boldsymbol{\Pr}}^{(a)} \big(X_{1} = \dots = X_{C_{a}} = a | N - C_1 - \cdots - C_{a-1} \big) \times \cdots }\\
   & {\color{white} \times {\boldsymbol{\Pr}}^{(A-1)} \big(X_{1} = \dots = X_{C_{A-1}} = A-1 | N - C_1 - \cdots - C_{A-2} \big) } \\
   & {\color{white} = \dfrac{q^{(1)}_{C_1,N}}{\binom{N}{C_1}} \times \dfrac{q^{(2)}_{C_2,N - C_1}}{\binom{N - C_1}{C_2}} \times \cdots \times \dfrac{q^{(a)}_{C_{a},N - C_1 - \cdots - C_{a-1}}}{\binom{N - C_1 - \cdots - C_{a-1}}{C_a}} \times \cdots \times \dfrac{q^{(A-1)}_{C_{A-1},N - C_1 - \cdots - C_{A-2}}}{\binom{N - C_1 - \cdots - C_{A-2}}{C_{A-1}}} }
\end{split}
\end{equation*}

\end{frame}
%---------------------------------------------------------

%---------------------------------------------------------
%Changing visivility of the text
\begin{frame}
\frametitle{Aim 3: Develop A Semi-parametric Model for Clustered Ordinal Outcomes}

\alert{Semi-parametric model}:
\begin{equation*}\label{M:SPforOrdinal2}
\begin{split}
    \boldsymbol{\Pr} \big( & X_1 = \dots = X_{C_1} = 1,  X_{C_1 + 1} = \dots = X_{C_1 + C_2} = 2, \cdots, \\
   & X_{C_1 + \dots + C_{a-1} + 1} = \dots = X_{C_1 + \cdots + C_{a-1} +  C_{a}} = a, \cdots, \\
   & X_{C_1 + \dots + C_{A-2} + 1} = \dots = X_{C_1 + \dots + C_{A-2} +  C_{A-1}} = A-1, \\
   & X_{C_1 + \dots + C_{A-1} + 1} = \dots = X_{C_1 + \dots + C_{A-1} +  C_{A}} = A | N \big) \\
   & = {\boldsymbol{\Pr}}^{(1)} \big( X_1 = \dots = X_{C_1} = 1 | N \big) \times {\boldsymbol{\Pr}}^{(2)} \big(X_{1} = \dots = X_{C_2} = 2 | N - C_1\big) \times \cdots \\
   & \times {\boldsymbol{\Pr}}^{(a)} \big(X_{1} = \dots = X_{C_{a}} = a | N - C_1 - \cdots - C_{a-1} \big) \times \cdots \\
   & \times {\boldsymbol{\Pr}}^{(A-1)} \big(X_{1} = \dots = X_{C_{A-1}} = A-1 | N - C_1 - \cdots - C_{A-2} \big) \\
   & {\color{white} = \dfrac{q^{(1)}_{C_1,N}}{\binom{N}{C_1}} \times \dfrac{q^{(2)}_{C_2,N - C_1}}{\binom{N - C_1}{C_2}} \times \cdots \times \dfrac{q^{(a)}_{C_{a},N - C_1 - \cdots - C_{a-1}}}{\binom{N - C_1 - \cdots - C_{a-1}}{C_a}} \times \cdots \times \dfrac{q^{(A-1)}_{C_{A-1},N - C_1 - \cdots - C_{A-2}}}{\binom{N - C_1 - \cdots - C_{A-2}}{C_{A-1}}} }
\end{split}
\end{equation*}

\end{frame}
%---------------------------------------------------------

%---------------------------------------------------------
%Changing visivility of the text
\begin{frame}
\frametitle{Aim 3: Develop A Semi-parametric Model for Clustered Ordinal Outcomes}

\alert{Semi-parametric model}:
\begin{equation*}\label{M:SPforOrdinal3}
\begin{split}
    \boldsymbol{\Pr} \big( & X_1 = \dots = X_{C_1} = 1,  X_{C_1 + 1} = \dots = X_{C_1 + C_2} = 2, \cdots, \\
   & X_{C_1 + \dots + C_{a-1} + 1} = \dots = X_{C_1 + \cdots + C_{a-1} +  C_{a}} = a, \cdots, \\
   & X_{C_1 + \dots + C_{A-2} + 1} = \dots = X_{C_1 + \dots + C_{A-2} +  C_{A-1}} = A-1, \\
   & X_{C_1 + \dots + C_{A-1} + 1} = \dots = X_{C_1 + \dots + C_{A-1} +  C_{A}} = A | N \big) \\
   & = {\boldsymbol{\Pr}}^{(1)} \big( X_1 = \dots = X_{C_1} = 1 | N \big) \times {\boldsymbol{\Pr}}^{(2)} \big(X_{1} = \dots = X_{C_2} = 2 | N - C_1\big) \times \cdots \\
   & \times {\boldsymbol{\Pr}}^{(a)} \big(X_{1} = \dots = X_{C_{a}} = a | N - C_1 - \cdots - C_{a-1} \big) \times \cdots \\
   & \times {\boldsymbol{\Pr}}^{(A-1)} \big(X_{1} = \dots = X_{C_{A-1}} = A-1 | N - C_1 - \cdots - C_{A-2} \big) \\
   & = \dfrac{q^{(1)}_{C_1,N}}{\binom{N}{C_1}} \times \dfrac{q^{(2)}_{C_2,N - C_1}}{\binom{N - C_1}{C_2}} \times \cdots \times \dfrac{q^{(a)}_{C_{a},N - C_1 - \cdots - C_{a-1}}}{\binom{N - C_1 - \cdots - C_{a-1}}{C_a}} \times \cdots \times \dfrac{q^{(A-1)}_{C_{A-1},N - C_1 - \cdots - C_{A-2}}}{\binom{N - C_1 - \cdots - C_{A-2}}{C_{A-1}}}
\end{split}
\end{equation*}

\end{frame}
%---------------------------------------------------------

%---------------------------------------------------------
%Changing visivility of the text
\begin{frame}
\frametitle{Aim 3: Develop A Semi-parametric Model for Clustered Ordinal Outcomes Cont.}

\begin{itemize}
    \item DMFT data from NHANES, patients records treated as random samples, which in fact based on complex, multi-stage design.
    \item \alert{Assumption}: probability density of smaller cluster of size $N'$ ($<N$) from exponentially tilted version of density for maximum cluster size $N$.
    \item $\left\lbrace q^{(a)}_{C_{a},N - C_1 - \cdots - C_{a-1}}, a=1, \dots, A-1 \right\rbrace$: marginal probability from smaller cluster of size $N' := {N - C_1 - \cdots - C_{a-1}}$.
    \item Subject to \alert{marginal compatibility} {\&} derive from exponentially tilted version of maximum cluster size's distribution $\left\lbrace q^{(a)}_{C_{a},N | N - C_1 - \cdots - C_{a-1}} \right\rbrace$.
    \item $q^{(a)}_{y,N'} =\binom{N'}{y} \displaystyle \sum_{t=y}^{N-N'+y} \dfrac{\binom{N-N'}{t-y} q^{(a)}_{t,N|N'}}{\binom{N}{t}}$.
\end{itemize}

\end{frame}
%---------------------------------------------------------

%---------------------------------------------------------
%Changing visivility of the text
\begin{frame}
\frametitle{Aim 3: Develop A Semi-parametric Model for Clustered Ordinal Outcomes Cont.}

\begin{itemize}
    \item \alert{Generalized marginal compatibility model for ordinal outcomes}:
    \begin{equation*}\label{M:GMCordinal_ExponentialTilting}
    \begin{split}
    q^{(a)}_{y,N|N'} &= q^{(a)}_{y,N} \times \exp{ \left[ - \omega^{(a)}_{N'} y - b \left( \omega^{(a)}_{N'} \right) \right]} \\
    & \propto q^{(a)}_{y,N} \times \exp{ \left( - \omega^{(a)}_{N'} y \right)} .
    \end{split}
    \end{equation*}
    \item $\omega^{(a)}_{N'}$: tilting parameter for cluster size $N'$ {\&} response category “$a$”.
    \item Similar Expectation Maximization Newton-Raphson algorithm \footcite{rathouz2008generalized} for parameter estimation.
    \item Additional penalty $\varrho_a \displaystyle\sum_{N'=1}^N (\omega_{N'} - \omega^{(a)}_{N'-1})^2$, $\omega^{(a)}_N=0$, assigned in Lagrangian.
    \item R package \textbf{CorrBin}.
\end{itemize}

\end{frame}
%---------------------------------------------------------



%---------------------------------------------------------
\begin{frame}[allowframebreaks]
     \printbibliography
 \end{frame}
%---------------------------------------------------------



\section{Acknowledgement}

%---------------------------------------------------------
\begin{frame}
     \begin{itemize}
     \item First of all, I would like to express my thanks to Dr. Szabo for her constant guidance throughout the proposal work and being my advisor. This proposal would not have been possible without her timely suggestion despite of her busy schedule. I also would like to express my deepest appreciation to my committee chair and advisor, Dr. Laud, for his continually and convincingly conveying a spirit of adventure in regard to research and scholarship, as well as an excitement in regard to teaching.
     \item  I will not forget to thank Dr. Wang, my RA advisor, whom for without his patience and guidance, I would not have made it so far, especially with my extended research in new developments on statistical machine learning.
     \item I also take this opportunity to specially thank Dr. Zhang, Dr. Banerjee and Dr. Okunseri for their valuable insights and kind help serving on my thesis committee. I would like extend my thanks to all the staff and faculty in the Division of Biostatistics.
     \item Finally, heartfelt gratitude to my friend, Ying, as well as family for the moral support and encouragement throughout graduation days.
     \end{itemize}
 \end{frame}
%---------------------------------------------------------
\end{document} 